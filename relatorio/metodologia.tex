\section{Metodologia}

Neste trabalho, estudaremos dois modelos para a dinâmica da infecção por HIV, obtidos de \cite{model-combined}.
O primeiro deles modela a evolução das populações de células T-CD4+ e de vírions enquanto não há tratamento.
O segundo assume que, em um dado tempo \( t_{ 0 } \), é iniciado tratamento combinado, feito com inibidores de transcriptase reversa e inibidores de protease.
Ele também admite que, em um tempo \( t_{ r } > t_{ 0 } \), a população viral se divide em duas, com o aparecimento de uma linhagem resistente ao tratamento.

\subsection{Modelo para período da doença sem tratamento}

As equações para o modelo sem tratamento são:
\begin{align}
    \dot{T}( t ) &= S ( t ) - \mu_{ T } T ( t ) + \frac{ p T ( t ) V_{ s } ( t ) }{ ( C + V_{ s } ( t ) ) } - k_{ s } V_{ s } ( t ) T ( t ) \label{Tponto_S}\\
    \dot{T_{ s }} ( t ) &= k_{ s } V_{ s } ( t ) T ( t ) - \mu_{ T_{ i } } T_{ s } ( t ) - \frac{ p_{ i } T_{ s } ( t ) V_{ s } ( t ) }{ ( C_{ i } + V_{ s } ( t ) ) } \label{Tsponto_S} \\
    \dot{V_{ s } ( t )} &= N p_{ i } T_{ s } ( t ) \frac{ V_{ s } ( t ) }{ ( C_{ i } + V_{ s } ( t ) ) } - k_{ \nu } T ( t ) V_{ s } ( t ) + \frac{ G_{ s } V_{ s } ( t ) }{ ( B + V_{ s } ( t ) ) } \label{Vsponto_S}
.\end{align}

Nessas equações, \( T ( t ) \) e \( T_{ s } ( t ) \) representam a população de células \( T \) saudáveis e de células \( T \) infectadas por uma linhagem do vírus suscetível a tratamento, respectivamente.
Denotamos a população suscetível do vírus por \( V_{ s } ( t ) \).
Todas essas populações são medidas em indivíduos por \( \unit{mm^{ 3 }} \) presentes no plasma sanguíneo, que representa \( 2 \% \) do total, sendo que o resto reside no tecido linforeticular \cite{model-combined}.

Na equação (\ref{Tponto_S}), o termo \( S ( t ) \) representa a produção de células \( T \) não infectadas pelo timo ou outros compartimentos do corpo.
Vamos assumir que há uma degeneração dessa fonte ao longo da infecção, partindo de \( 10 \ \unit{mm^{ -3 } dia^{ -1 }} \) e decrescendo com o aumento da população viral, até o mínimo de \( 3 \ \unit{mm^{ -3 } dia^{ -1 }} \).
Esse decaimento é modelado como a seguir:
\begin{equation}
    S ( t ) = \left(
        10  - 7 \frac{ V ( t ) }{ ( B_{ s } + V ( t ) ) } 
    \right)
    \unit{mm^{ -3 } dia^{ -1 }}
.\end{equation}
Aqui, \( B_{ s } \) é uma constante de saturação e \( V ( t ) \) é a população total viral.
Enquanto não há tratamento, essa população é igual a \( V_{ s } ( t ) \), porém, uma vez que o tratamento se iniciar e aparecer uma população resistente de vírus, \( V ( t ) \) será a soma das duas.
Para poder ser somada a \( V ( t ) \), a unidade de medida de \( B_{ s } \) deve ser \( \unit{mm^{ -3 }} \).

Ainda na primeira equação, \( \mu_{ T } \) é a taxa de mortalidade das células \( T \) não infectadas, \( p \) é uma taxa que mede o ganho de células \( T \) em função da divisão celular provocada pelo contato com o antígeno, \( C \) é uma constante de saturação e \( k_{ s } \) é a taxa de infeção das células saudáveis pelo vírus.
O termo responsável pela infecção de novas células segue a lei de ação de massas.
Como o termo \( \mu_{ T } T ( t ) \) deve estar em \( \unit{mm^{ -3 } dia^{ -1 }} \), a unidade de medida de \( \mu_{ T } \) é \( \unit{dia^{ -1 }} \).
De maneira análoga \( B_{ s } \), a unidade de \( C \) é \( \unit{mm^{ -3 }} \).
Multiplicando as unidades de \( T ( t ) V_{ s } ( t ) / ( C + V_{ s } ( t ) ) \), obtemos \( \unit{mm^{ -3 }} \), logo, para a unidade final ser \( \unit{mm^{ -3 } dia^{ -1 }} \), \( p \) deve ser medido em \( \unit{dia^{ -1 }} \).


Na equação (\ref{Tsponto_S}) há, naturalmente, um ganho de células infectadas dado pelo termo \( k_{ s } V_{ s } ( t ) T ( t ) \), bem como uma taxa de mortalidade dessas células.
Devido ao estresse causado pela presença do vírus, tomaremos \( \mu_{ T_{ i } } > \mu_{ T } \).
Além disso, o termo \( p_{ i } T_{ s } ( t ) V_{ s } ( t ) / ( C_{ i } + V_{ s } ( t ) ) \) representa a lise (ruptura da membrana) celular, causada pela quantidade excessiva de vírus invasores.
Quando se rompe dessa forma, a célula libera os vírions que estavam em seu interior.

Na equação (\ref{Vsponto_S}), o termo \( N p_{ i } T_{ s } ( t ) V_{ s } ( t ) / ( C_{ i } + V_{ s } ( t ) ) \) corresponde ao aumento da população viral provocado pelos vírions liberados na lise de células \( T \) contaminadas.
A constante \( N \) é o número de vírions liberados por cada célula.
O termo \( k_{ \nu } T ( t ) V_{ s } ( t ) \) representa a resposta imune, novamente seguindo a lei de ação de massas.
Por fim, modelamos a fonte de vírus externa ao sangue (baço e nodos linfáticos, por exemplo) por meio do termo \( G_{ s } V_{ s } ( t ) / ( B + V_{ s } ( t ) ) \), onde \( B \) é uma constante de saturação.
O parâmetro \( G_{ s } \) ditará a influência que o ambiente externo ao sangue tem na dinâmica da infecção, sendo o principal responsável por determinar o comportamento global da doença, ou seja, se o quadro do hospedeiro evoluirá para a extinção do vírus, uma população estável de HIV ou progredirá para AIDS.
Essa influência pode ser parcialmente explicada quando lembramos que cerca de \( 98 \% \) das populações do vírus e de céluas \( T \) não se encontra no sangue.

\subsection{Modelo para período da doença em que há tratamento}

Supomos que até um tempo \( t_{ 0 } \), a evolução da infecção é descrita pelas equações (\ref{Tponto_S}), (\ref{Tsponto_S}) e (\ref{Vsponto_S}), pois não há tratamento.
No tempo \( t_{ 0 } \), é iniciado tratamento e, em um tempo \( t_{ r } > t_{ 0 } \), surge uma linhagem viral resistente ao tratamento.
A partir de \( t_{ 0 } \), as equações que goveranm o modelo são
\begin{align}
    \dot{T}( t ) &= S_{ 0 } ( t ) - \mu_{ T } T ( t ) + \frac{ p T ( t ) V ( t ) }{ ( C + V ( t ) ) } - ( \mu k_{ s } V_{ s } ( t ) + k_{ k } V_{ r } ( t ) ) T ( t ) \label{Tponto_T}\\
    \dot{T_{ s }} ( t ) &= \mu k_{ s } V_{ s } ( t ) T ( t ) - \mu_{ T_{ i } } T_{ s } ( t ) - \frac{ p_{ i } T_{ s } ( t ) V ( t ) }{ ( C_{ i } + V ( t ) ) } \label{Tsponto_T} \\
    \dot{ T_{ r } } &= k_{ r } V_{ r } ( t ) T ( t ) - \mu_{ T_{ i } } T_{ r } ( t ) - \frac{ p_{ i } T_{ r } ( t ) V ( t ) }{ C_{ i } + V ( t ) } \label{Trponto_T} \\
    \dot{V_{ s } } ( t ) &= \frac{ \rho q ( t ) N p_{ i } T_{ s } ( t ) V ( t ) }{ C_{ i } + V ( t ) } + \frac{ ( 1 - q ( t ) ) N p_{ i } T_{ r } ( t ) V ( t ) }{ C_{ i } + V ( t ) } - k_{ \nu } T_{ t } V_{ s } ( t ) + \eta \frac{ G_{ s } V_{ s } ( t ) }{ B + V ( t ) } \label{Vsponto_T} \\
    \dot{ V_{ r }} ( t ) &= \frac{ q ( t ) N p_{ i } T_{ r } ( t ) V ( t ) }{ C_{ i } + V ( t ) } + \frac{ ( 1 - q ( t ) ) N p_{ i } T_{ s } ( t ) V ( t ) }{ C_{ i } + V ( t ) } - k_{ \nu } T ( t ) V_{ r } ( t ) + \frac{ G_{ r } V_{ r } ( t ) }{ B + V ( t ) } \label{Vrponto_T}
.\end{align}
Os valores iniciais \( V_{ s } ( t_{ 0 } ), T ( t_{ 0 } ) \) e \( T_{ s } ( t_{ 0 } ) \) para essa segunda etapa do modelo são obtidos utilizando as equações para o caso sem tratamento.
Além disso, naturalmente temos \( V_{ r } ( t_{ 0 } ) = T_{ r } ( t_{ 0 } ) = 0 \).

O tratamento é incorporado por meio de uma redução de alguns termos relacionados ao espalhamento do vírus.
Nas equações (\ref{Tponto_T}) e (\ref{Tsponto_T}), os termos que representam a infecção de células T saudáveis estão multiplicados por uma constante \( \mu < 1 \).
Isso simula o efeito de um inibidor de transcriptase reversa, o qual reduz a infecção de novas células, porém não interrompe a produção viram em células já contaminadas.
Por outro lado, na equação (\ref{Vsponto_T}), o termo que representa o ganho de vírions provenientes da lise de células T contaminadas é multiplicado por uma constante \( \rho < 1 \).
Isso corresponde à ação de um inibidor de protease, que age na fase do ciclo de vida viral que ocorre dentro da célula, bloqueando a produção de novas partículas virais.
Por fim, a fonte viral externa na equação (\ref{Vsponto_T}) também sofre uma penalidade dada pelo fator \( \eta < 1 \), representando o efeito do tratamento sobre a replicação dos vírus que não se encontram na corrente sanguínea.
Observe que os efeitos do tratamento só se aplicam aos termos relacionados à população viral suscetível.

Em um dado tempo \( t_{ r } > t_{ 0 } \) surge uma variante viral resistente ao tratamento.
Antes dele, pressupomos que não há quantidades significativas de vírus resistentes.
Após \( t_{ r } \), consideramos que uma parcela \( q < 1 \) do vírus suscetível que é produzido se mantém suscetível, a parcela restante, \( 1 - q \), sofre uma mutação e se torna resistente.
Nas equações (\ref{Vsponto_T}) e (\ref{Vrponto_T}), isso é incorporado por meio da função \( q ( t ) \), dada por \( q ( t ) = 1 \) se \( t < t_{ r } \) e \( q ( t ) = q \) se \( t > t_{ r } \).
Supomos que o fator que determina a fonte externa de vírus resistente, \( G_{ r } \), é ligeiramente menor que seu correspondente para o vírus suscetível, \( G_{ s } \).
