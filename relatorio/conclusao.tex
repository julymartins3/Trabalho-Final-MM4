\section{Conclusão}

Os modelos apresentados produzem simulações que são condizentes com o que é observado na vida real, e sugerem uma estratégia de tratamento combinado que pode levar a uma maior longevidade do paciente, e.g., iniciar o tratamento quando a concentração de células T no sangue atingir um nível intermediário.
Entretanto, supondo um quadro de progressão à AIDS, não é possível evitar o óbito,
devido ao aparecimento de variantes de HIV resistentes ao tratamento.

A principal limitação do modelo é o fato dele se atentar apenas às populações viral e de células T no sangue, sendo que elas correspondem a uma pequena fração das populações corporais.
Em próximos trabalhos, é necessário modelar, também, como evoluem essas populações em outros compartimentos do corpo.

Também seria interessante reavaliar algumas hipóteses feitas, como, por exemplo, a hipótese de que a produção de células T se torna limitada superiormente a partir do início do tratamento.
Outra possibilidade seria explorar os efeitos que outras drogas podem ter no curso da doença, além da combinação de inibidores de protease e de inibidores de transcriptase reversa.
