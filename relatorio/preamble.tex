%% Packages
\usepackage{mathtools, amsthm}
% \usepackage{tikz}
% \usetikzlibrary{positioning}
% \usepackage{csquotes}
% \usepackage{hyperref}

%% Environments

\theoremstyle{plain} % default
\newtheorem{teo}{Teorema}[section]
\newtheorem*{teo*}{Teorema}
\newtheorem{lem}{Lema}[section]
\newtheorem{prop}{Proposição}[section]
\newtheorem{cor}[teo]{Corolário}
\newtheorem*{axiom}{Axioma}

\newtheorem*{TAU}{Teorema da Aproximação Universal}
\newtheorem*{Riesz}{Teorema da Representação de Riesz}

\theoremstyle{definition}
\newtheorem{defn}{Definição}[section]
\newtheorem{conj}{Conjectura}[section]
\newtheorem{exmp}{Exemplo}[section]
\newtheorem{rem}{Observação}[section]
\newtheorem*{rem*}{Observação}

\theoremstyle{remark}
\newtheorem*{note}{Nota}
\newtheorem{case}{Caso}


% Macros

\renewcommand{\vec}[1]{\mathbf{#1}}
\renewcommand{\Re}{\text{Re}}

\newcommand{\K}{\mathbb{K}}
\newcommand{\I}{\mathbb{I}}

\DeclarePairedDelimiter{\dotprod}{\langle}{\rangle}

\DeclareMathOperator{\rk}{rk}
\DeclareMathOperator{\intt}{int}
\DeclareMathOperator{\diam}{diam}
\DeclareMathOperator{\rref}{rref}
\DeclareMathOperator{\vspan}{span}
\DeclareMathOperator{\proj}{proj}
\DeclareMathOperator{\lin}{Lin}
\DeclareMathOperator{\supp}{supp}

\newcommand{\func}[3]{#1 : #2 \rightarrow #3}
\newcommand{\R}{\mathbb{R}}
\newcommand{\Z}{\mathbb{Z}}
\newcommand{\N}{\mathbb{N}}
\newcommand{\Q}{\mathbb{Q}}
\newcommand{\rr}{R_{r}}
\newcommand{\tq}{ : }
\newcommand{\mdc}{\text{mdc}}
\newcommand{\mmc}{\text{mmc}}
\newcommand{\defeq}{\vcentcolon=}
\newcommand{\comp}{\mathscr{C}}


%% Upper and Lower Integrals
\newcommand{\loint}[4]{
    \lefteqn{\int_{ #1 }^{ #2 } #3}\lefteqn{\hspace{0.0ex}\rule[-2.25ex]{1.1ex}{.05ex}} \phantom{\int_{ #1 }^{ #2 } #3}\mathrm{d}#4
}
\newcommand{\upint}[4]{
    \lefteqn{\int_{ #1 }^{ #2 } #3 \ }\lefteqn{\hspace{1.2ex}\rule[ 3.35ex]{1.1ex}{.05ex}} \phantom{\int_{ #1 }^{ #2 } #3 \ }\mathrm{d}#4
}

\DeclarePairedDelimiter\ceil{\lceil}{\rceil}
\DeclarePairedDelimiter\floor{\lfloor}{\rfloor}
\DeclarePairedDelimiter\abs{\lvert}{\rvert}%
\DeclarePairedDelimiter\norm{\lVert}{\rVert}%
\DeclareMathOperator{\sen}{sen}

% Swap the definition of \abs* and \norm*, so that \abs
% and \norm resizes the size of the brackets, and the 
% starred version does not.

\makeatletter
\let\oldabs\abs
\def\abs{\@ifstar{\oldabs}{\oldabs*}}
%
\let\oldnorm\norm
\def\norm{\@ifstar{\oldnorm}{\oldnorm*}}
\makeatother

\newcommand{\transpose}{\mathsf{T}}
