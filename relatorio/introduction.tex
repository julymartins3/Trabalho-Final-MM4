\section{Introdução}

%% Descrever brevemente o que é o vírus HIV e a AIDS.

%% Falar sobre como o vírus interage com células CD4+.

%% Falar sobre tratamento para HIV e os desafios (aparecimento de
 % variantes resistentes).

%% Breve resumo de outros trabalhos.

Na década de 80, foi descoberto o vírus da imunodeficiência humana (HIV -- \textit{humam immunodeficiency virus}), cuja infecção é uma das principais causas de morbidade e mortalidade ao redor do mundo \cite{nature}.
A doença causada pelo vírus é provocada pela morte de células T-CD4+, as quais são uma parte vital do sistema imunológico, responsáveis por regular a resposta do corpo a um patógeno, ativando outras células do sistema imunológico ou, caso necessário, suprimindo a reação imune \cite{tcd4+}.

A infeção por HIV possui algumas etapas bem definidas.
Inicialmente, logo após infectar o hospedeiro, o vírus ocupa os tecidos mucosos e, em alguns dias, se espalha para os órgãos linfoides.
Ele, então, se multiplica exponencialmente e tem um pico de concentração por volta do dia 30, quando os níveis de anticorpos ficam detectáveis.
Nesse momento, o sistema imunológico obtém um certo controle, e a infecção atinge um nível estável que, geralmente, se mantém por anos.
Nesse tempo, o HIV provoca uma queda constante de células T-CD4+ e, eventualmente, uma imunodeficiência profunda se apresenta, fazendo com que o hospedeiro desenvolva complicações infecciosas ou oncológicas, que definem a AIDS \cite{nature}.

O tratamento para infecção por HIV é denominado ART (\textit{antiretroviral drug therapy}) e está disponível há mais de duas décadas.
Quando se demonstra eficaz, provoca uma supressão completa da replicação viral, diminuindo drasticamente a chance do quadro de infecção evoluir para AIDS \cite{nature}.
Evidências demonstraram repetidamente que a eficácia do tratamento feito com o uso de dois ou mais agentes antiretrovirais é significativamente maior \cite{comb-drug}.

Há, na literatura, uma série de artigos que modelam a interação entre o vírus HIV e o sistema imunológico.
Por exemplo, \cite{model-simple} modela a dinâmica global da infecção de HIV, utilizando um termo de crescimento logístico para corresponder à produção extra de céluas T-CD4+ em função do contato com o vírus.
Em uma outra abordagem, \cite{model-treatment} modela o curso da infecção considerando que, em algum momento, é iniciada uma monoterapia com um inibidor de transcriptase reversa.
Neste trabalho, entretanto, exploraremos um modelo que incorpora o tratamento utilizando uma combinação de dois tipos de drogas: o já citado inibidor da transcriptasee um inibidor de protease.
A principal referência para esse tipo de modelo é \cite{model-combined}.


